\documentclass[a4paper,margin=1in]{article}

%— Page layout
\usepackage{geometry}
\usepackage{float}                       % [H] placement option

%— Fonts & symbols
\usepackage{times}                       % Times font
\usepackage{textcomp}                    % additional text symbols
\usepackage{bm}                          % bold math
\usepackage{bbm}                         % blackboard‑bold math
\usepackage{gensymb}                     % degree symbol, etc.

%— Mathematics
\usepackage{mathtools}                   % loads amsmath + extras
\usepackage{amssymb,amsfonts}            % symbols & fonts
\usepackage{mathrsfs}                    % script fonts

%— Theorems & citations
\usepackage{thmtools}                    % enhanced theorem environments
\usepackage{cite}                        % sorted numeric citations

%— Graphics & plots
\usepackage[dvipsnames]{xcolor}
%\usepackage{graphicx}                    % \includegraphics
\usepackage{subcaption}                  % sub‑figures
\usepackage{tikz}                        % drawing
\usepackage{pgfplots}                    % plotting
\pgfplotsset{compat=1.17}                % adjust to your version

%— Color & hyperlinks
\usepackage[dvipsnames]{xcolor}          % named colors
\usepackage[colorlinks,
            linkcolor=blue,
            citecolor=blue,
            urlcolor=blue]{hyperref}

%— Tables & arrays
\usepackage{array}                       % array extensions
\usepackage{blkarray}                    % block‑array annotations

%— Footnotes & captions
\usepackage[hang,flushmargin]{footmisc}  % styled footnotes
\usepackage[justification=centering,
            skip=5pt]{caption}           % caption spacing

%— Optional TikZ libraries (load only what you use)
\usetikzlibrary{arrows,automata,spy,fillbetween}
\usepgfplotslibrary{external,fillbetween}

%— URL handling
\usepackage{url}
%— Symbols
\usepackage{textcomp}       % extra text symbols (\texteuro, etc.)
\usepackage{gensymb}        % degree symbol, etc.

%— Mathematics
\usepackage{mathtools}      % loads amsmath + useful extensions

%— Graphics & Floats
%\usepackage{graphicx}       % \includegraphics
\usepackage{float}          % [H] float placement

%— Figure captions & sub‑figures
\usepackage[justification=centering,
            skip=5pt]{caption}      % caption spacing
\usepackage{subcaption}               % sub‑figure environments

\input{macros}


\begin{document}

\noindent Professor Alex Spector\\
EN.585.743.81.SP25 Modeling Approaches to Cell and Tissue Engineering\\
Modeling Project\\
Johns Hopkins University\\
Student: Yves Greatti\\\

\section*{Introduction}
In Lecture 12, the polarization of the stem cell cytoskeleton as a function of
the stiffness of ECM is considered. A number of characteristics of cell/ECM
interaction, such as the modeling-predicted order parameter \cite{Zemel2010}
(Figure~\ref{fig:combined}(a)) or experiment-estimated myosin fiber intensity
\cite{Zemel2010} (Figure~\ref{fig:combined}(b)) demonstrate an
increase–saturation pattern as functions of ECM stiffness.
Use the 1-D model where the cell and ECM are presented by the active and
elastic springs, respectively \cite{Zemel2010}  (Figure~\ref{fig:1d-spring-model}) and show that the active force generated
by the cell, f$_a$, as a function of the ECM stiffness, has the same increase 
saturation pattern of behavior.


\begin{figure}[h!]
  \centering
  \begin{subfigure}[b]{0.45\textwidth}
    \includegraphics[width=\textwidth]{order_parameter}
    \caption{Normalized order parameter vs.\ \(\frac{E_m}{E_c}\).}
    \label{fig:order_parameter}
  \end{subfigure}
  \quad
  \begin{subfigure}[b]{0.45\textwidth}
    \includegraphics[width=\textwidth]{actomyosin}
    \caption{Myosin fibre intensity vs.\ \(E_m\).}
    \label{fig:myosin}
  \end{subfigure}
  \caption{(a) Order–parameter saturation. (b) Actomyosin‐intensity saturation.}
  \label{fig:combined}
\end{figure}

\section*{Derivation}

We want to show, using the 1D two‐spring model of cell and matrix (Figure~\ref{fig:1d-spring-model}),
that the cell’s active force \(f_a\) satisfies the equation:
\[
\frac{f^{a}}{f^{0}} = \alpha \,\frac{k_{m}}{\tilde{k}_{c} + k_{m}}
\]
At equilibrium:
\begin{align}
  k_c (l_c - l_c^R) &= k_m (l_m - l_m^R)  \label{eq:force-balance}\\
  l_c + l_m         &= l_c^0 + l_m^R      \label{eq:length-sum}
\end{align}

From \eqref{eq:length-sum}:
\begin{equation} \label{eq:lm}
  l_m = (l_c^0 + l_m^R) - l_c
\end{equation}

Substitute (3) into the force-balance \eqref{eq:force-balance}:
\begin{align*}
  k_c (l_c - l_c^R)
  &= k_m \bigl(l_m - l_m^R\bigr) \nonumber\\
  &= k_m\bigl((l_c^0 + l_m^R) - l_c - l_m^R\bigr)\nonumber\\
  &= k_m (l_c^0 - l_c)
\end{align*}

Rearrange to solve for \(l_c\):
\begin{align*}
  k_c\,l_c - k_c\,l_c^R       &= k_m\,l_c^0 - k_m\,l_c \nonumber\\
  k_c\,l_c + k_m\,l_c         &= k_m\,l_c^0 + k_c\,l_c^R \nonumber\\
  (k_c + k_m)\,l_c            &= k_m\,l_c^0 + k_c\,l_c^R \nonumber\\
  l_c                         &= \frac{k_m\,l_c^0 + k_c\,l_c^R}{k_c + k_m}
\end{align*}

We have
\[
  l_c^R = l_c^0 + \Delta l_c^0
\]
then
\begin{align*}
  l_c
  &= \frac{k_m\,l_c^0 + k_c\,(l_c^0 + \Delta l_c^0)}{k_c + k_m}\\
  &= \frac{(k_m + k_c)\,l_c^0 + k_c\,\Delta l_c^0}{k_c + k_m}\\
  &= l_c^0 + \frac{k_c}{k_c + k_m}\,\Delta l_c^0
\end{align*}
so
\[
  \Delta l_c \;=\; l_c - l_c^0 \;=\; \frac{k_c}{k_c + k_m}\,\Delta l_c^0
\]

Introducing myosin polarization of the cytoskeleton fibers within the cell,  with 
\[
  \tilde{k}_c \;=\; (1 + \alpha)\,k_c \quad (\alpha>0)
\]
the cellular strain satisfies the equation
\[
  \frac{\Delta l_c}{l_c^0}
  = \frac{\tilde{k}_c}{\tilde{k}_c + k_m}\,\frac{\Delta l_c^0}{l_c^0}
\]

The active force is modeled as:
\[
  f^a = -\,\alpha\,k_c\,(l_c - l_c^R)
\]
from
\[
  \Delta l_c \;=\; l_c - l_c^0 \;=\; \,\frac{\tilde{k}_c}{\tilde{k}_c + k_m} \,\Delta l_c^0
\]

Thus:
\begin{align*}
l_c^R - l_c &= (l_c^R - l_c^0) + (l_c^0 - l_c) \\
&= \Delta l_c^0 - \frac{\tilde{k}_c}{\tilde{k}_c + k_m} \Delta l_c^0 \\
&= \left(1 - \frac{\tilde{k}_c}{\tilde{k}_c + k_m} \right) \Delta l_c^0 \\
&= \frac{k_m}{\tilde{k}_c + k_m} \Delta l_c^0
\end{align*}

Substituting back into the active force:
\[
f^a = -\alpha k_c (l_c - l_c^R) = \alpha k_c \left( \frac{k_m}{\tilde{k}_c + k_m} \Delta l_c^0 \right)
\]

Since the baseline force \(f^0\) is proportional to \(k_c \Delta l_c^0\):
\[
f^0 = k_c \Delta l_c^0
\]
Thus, we have:
\[
\boxed{
\frac{f^a}{f^0} = \alpha \, \frac{k_m}{\tilde{k}_c + k_m}
}
\]

\begin{figure}[h!]
  \centering
  \includegraphics[width=1\textwidth]{springs}
  \caption{1D model: active cell spring (stiffness \(k_c\)) in series with 
  ECM spring (stiffness \(k_m\)).}
  \label{fig:1d-spring-model}
\end{figure}

\section*{Interpretation}

The formal form of anisotropic polarized actomyosin force is:
\[
  \frac{f^{a}}{f^{0}}
  = \alpha \,\frac{k_{m}}{\tilde{k}_{c} + k_{m}}
\]

\begin{itemize}
  \item $\displaystyle \alpha$: polarizability factor
  \item $\displaystyle \tilde{k}_{c} = (1 + \alpha)\,k_{c}$: effective stiffness of the cell
  \item $k_{m}$: matrix rigidity
\end{itemize}

\begin{itemize}
  \item For very soft matrices (\(k_{m}\ll \tilde{k}_{c}\)):
    \[
      \frac{f^{a}}{f^{0}}
      = \alpha \,\frac{k_{m}}{\tilde{k}_{c} + k_{m}}
      \;\longrightarrow\;0
      \quad\text{as }k_{m}\to 0.
    \]
    If \(k_{m}\ll \tilde{k}_{c}\), then \(\tilde{k}_{c}+k_{m}\approx \tilde{k}_{c}\) and
    \[
      \frac{f^{a}}{f^{0}}
      \approx \alpha \,\frac{k_{m}}{\tilde{k}_{c}}
       \approx   \alpha^{*}  \,k_{m}
      \;\longrightarrow\;0
      \quad\text{as }k_{m}\to 0.
    \]
  \item For very stiff matrices (\(k_{m}\gg \tilde{k}_{c}\)):
    \[
      \frac{f^{a}}{f^{0}}
      = \alpha \,\frac{k_{m}}{\tilde{k}_{c} + k_{m}}
      \approx \alpha \,\frac{k_{m}}{k_{m}} 
      \;\longrightarrow\;\alpha
    \] 
\end{itemize}

Thus, the active force grows with matrix stiffness and  
saturates at a maximum value proportional to~$\alpha$.

\begin{figure}[H]
  \centering
  \includegraphics[width=0.8\textwidth]{plot}
  \caption{Active Force vs. Matrix stiffness.}
  \label{fig:activeForcet}
\end{figure}
     
\bibliographystyle{plain}
\bibliography{bibliography}

\end{document}
