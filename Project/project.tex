%\documentclass[12pt,twoside]{article}
\documentclass{article}
\usepackage[a4paper,margin=1in]{geometry}

\usepackage[dvipsnames]{xcolor}
\usepackage{tikz,graphicx,amsmath,amsfonts,amscd,amssymb,mathrsfs, bm,cite,epsfig,epsf,url}
\usepackage[hang,flushmargin]{footmisc}
\usepackage[colorlinks=true,urlcolor=blue,citecolor=blue]{hyperref}
\usepackage{amsthm,multirow,wasysym,appendix}
\usepackage{array,subcaption} 
% \usepackage[small,bf]{caption}
\usepackage{bbm}
\usepackage{pgfplots}
\usetikzlibrary{spy}
\usepgfplotslibrary{external}
\usepgfplotslibrary{fillbetween}
\usetikzlibrary{arrows,automata}
\usepackage{thmtools}
\usepackage{blkarray} 
\usepackage{textcomp}
\usepackage{float}
%\usepackage[left=0.8in,right=1.0in,top=1.0in,bottom=1.0in]{geometry}


\usepackage{times}
\usepackage{amsfonts}
\usepackage{amsmath}
\usepackage{latexsym}
\usepackage{color}
\usepackage{graphics}
\usepackage{enumerate}
\usepackage{amstext}
\usepackage{blkarray}
\usepackage{url}
\usepackage{epsfig}
\usepackage{bm}
\usepackage{hyperref}
\hypersetup{
    colorlinks=true,
    linkcolor=blue,
    filecolor=magenta,      
    urlcolor=blue,
}
\usepackage{textcomp}
%\usepackage[left=0.8in,right=1.0in,top=1.0in,bottom=1.0in]{geometry}
\usepackage{mathtools}
%\usepackage{minted}
\usepackage{gensymb}

\usepackage{graphicx}
\usepackage{float}
\usepackage[justification=centering,singlelinecheck=true]{caption}
\usepackage{subcaption}   
\usepackage{caption}
\captionsetup[figure]{skip=5pt}   % ← distance between figure and caption

\input{macros}


\begin{document}

\noindent Professor Alex Spector\\
EN.585.743.81.SP25 Modeling Approaches to Cell and Tissue Engineering\\
Modeling Project\\
Johns Hopkins University\\
Student: Yves Greatti\\\

\section*{Introduction}
In Lecture 12, the polarization of the stem cell cytoskeleton as a function of
the stiffness of ECM is considered. A number of characteristics of cell/ECM
interaction, such as the modeling-predicted order parameter \cite{Zemel2010}
(Figure~\ref{fig:combined}(a)) or experiment-estimated myosin fiber intensity
\cite{Zemel2010} (Figure~\ref{fig:combined}(b)) demonstrate an
increase–saturation pattern as functions of ECM stiffness.
Use the 1-D model where the cell and ECM are presented by the active and
elastic springs, respectively \cite{Zemel2010}  (Figure~\ref{fig:1d-spring-model}) and show that the active force generated
by the cell, f$_a$, as a function of the ECM stiffness, has the same increase 
saturation pattern of behavior.


\begin{figure}[h!]
  \centering
  \begin{subfigure}[b]{0.45\textwidth}
    \includegraphics[width=\textwidth]{order_parameter}
    \caption{Normalized order parameter vs.\ \(\frac{E_m}{E_c}\).}
    \label{fig:order_parameter}
  \end{subfigure}
  \quad
  \begin{subfigure}[b]{0.45\textwidth}
    \includegraphics[width=\textwidth]{actomyosin}
    \caption{Myosin fibre intensity vs.\ \(E_m\).}
    \label{fig:myosin}
  \end{subfigure}
  \caption{(a) Order–parameter saturation. (b) Actomyosin‐intensity saturation.}
  \label{fig:combined}
\end{figure}


\section*{Anisotropic Polarized Actomyosin Force}

\[
  \frac{f^{a}}{f^{0}}
  = \alpha \,\frac{k_{m}}{\tilde{k}_{c} + k_{m}}
\]

\begin{itemize}
  \item $\displaystyle \alpha$: polarizability factor
  \item $\displaystyle \tilde{k}_{c} = (1 + \alpha)\,k_{c}$: effective stiffness of the cell
  \item $k_{m}$: matrix rigidity
\end{itemize}

\section*{Interpretation}

\begin{itemize}
  \item For very soft matrices (\(k_{m}\ll \tilde{k}_{c}\)):
    \[
      \frac{f^{a}}{f^{0}}
      = \alpha \,\frac{k_{m}}{\tilde{k}_{c} + k_{m}}
      \;\longrightarrow\;0.
    \]
    If \(k_{m}\ll \tilde{k}_{c}\), then \(\tilde{k}_{c}+k_{m}\approx \tilde{k}_{c}\) and
    \[
      \frac{f^{a}}{f^{0}}
      \approx \alpha \,\frac{k_{m}}{\tilde{k}_{c}}
       \approx  \,k_{m}
      \;\longrightarrow\;0
      \quad\text{as }k_{m}\to 0.
    \]
  \item For very stiff matrices (\(k_{m}\gg \tilde{k}_{c}\)):
    \[
      \frac{f^{a}}{f^{0}}
      = \alpha \,\frac{k_{m}}{\tilde{k}_{c} + k_{m}}
      \approx \alpha \,\frac{k_{m}}{k_{m}} 
      \;\longrightarrow\;\alpha
    \] 
\end{itemize}

Thus, the active force grows with matrix stiffness and  
saturates at a maximum value proportional to~$\alpha$.

\section*{Derivation.}

Here we show, using the 1-D two‐spring model of cell and matrix  (Figure~\ref{fig:1d-spring-model}),
that the cell’s active force \(f_a\) vs.\ ECM stiffness \(k\) follows the
same functional form:

\begin{figure}[h!]
  \centering
  \includegraphics[width=1\textwidth]{springs}
  \caption{1D model: active cell spring (stiffness \(k_c\)) in series with \\
  ECM spring (stiffness \(k_m\)).}
    \label{fig:1d-spring-model}
\end{figure}


\section*{Illustration}

\begin{figure}[H]
  \centering
  \includegraphics[width=0.8\textwidth]{plot}
  \caption{Active Force vs. Matrix stiffness.}
  \label{fig:activeForcet}
\end{figure}
     
\bibliographystyle{plain}
\bibliography{bibliography}

\end{document}
